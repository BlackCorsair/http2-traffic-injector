\newpage
\section{Introducción}
El proyecto consiste en desarrollar mediante la metodología \emph{Extreme Programming}\cite{xp}, un inyector de tráfico HTTP2\cite{http2} que permita la automatización de pruebas \emph{no funcionales} en microservicios que exponen una API REST a través del protocolo HTTP2.

Este proyecto estará desarrollado en C++20\cite{cpp}, aprovechando la librería \emph{nghttp}\cite{nghttp}.


\section{Objetivos}
\subsection{Generales}
Con este proyecto se espera alcanzar los siguientes objetivos:

\begin{itemize}
    \item Inyector de tráfico HTTP2
    \item Pruebas asociadas
    \item Entorno de Integración Contínua
    \item Documentación necesaria
    \item Profundizar en la metodología \emph{Extreme Programming}
\end{itemize}

\subsection{Específicos}
\begin{itemize}
    \item Inyector de tráfico HTTP2
    \begin{itemize}
        \item Inyectar tráfico con operaciones simples
        \item Inyectar tráfico con operaciones dobles
        \item Ejecución mediante consola de comandos
        \item Configuración del tráfico mediante ficheros \emph{yaml}
    \end{itemize}
    \item Pruebas asociadas
    \begin{itemize}
        \item Pruebas unitarias
        \item Pruebas End to End
    \end{itemize}
    \item Entorno de Integración Contínua
    \begin{itemize}
        \item Pipelines que permitan la aprobación del commit en el repositorio una vez pasado con éxito los tests
        \item Pipelines que promocionen y versionen los artefactos generados
    \end{itemize}
    \item Documentación necesaria
    \begin{itemize}
        \item Modelo de vistas 4+1
        \item Utilización del producto
        \item Cómo cotribuir y utilizar el desarrollo de Integración Contínua
    \end{itemize}
\end{itemize}

\section{Metodología}
La metodología utilizada será \emph{Extreme Programming}.


\newpage
\begin{thebibliography}{Bibliografía}
\bibitem{xp}
Extreme Programming
\\\url{https://www.tutorialspoint.com/extreme_programming/extreme\_programming\_introduction.htm}

\bibitem{cpp}
C++20
\\\url{https://en.cppreference.com/w/cpp/20}

\bibitem{http2}
HTTP2
\\\url{https://http2.github.io/}

\bibitem{nghttp}
nghttp
\\\url{https://http2.github.io/}
\end{thebibliography}