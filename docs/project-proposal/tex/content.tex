\newpage
\section{Introducción}
El proyecto consiste en desarrollar mediante la metodología \emph{Extreme Programming}\cite{xp}, un inyector de tráfico HTTP2\cite{http2} que permita la automatización de pruebas \emph{no funcionales} en microservicios que exponen una API REST a través del protocolo HTTP2.

Este proyecto estará desarrollado en C++20\cite{cpp}, aprovechando la librería \emph{nghttp2}\cite{nghttp2}.

\section{Objetivos}
\subsection{Generales}
Con este proyecto se espera alcanzar los siguientes objetivos:
\begin{itemize}
    \item Un inyector de tráfico HTTP2 completamente funcional que permita
    \item Pruebas asociadas en todas las fases del desarrollo
    \item Alta calidad del codigo en todas las fases del desarrollo
    \item Profundizar en la metodología \emph{Extreme Programming}
\end{itemize}

\subsection{Específicos}
\begin{itemize}
    \item Para profundizar en la metodología \emph{Extreme Programming} se plantea el desarrollo en 2 fases:
    \begin{itemize}
        \item Primera fase en la que se plantea una funcionalidad sencilla y las pruebas del sistema
        \item Segunda fase en la que se plantea una funcionalidad avanzada y se demuestra que se alcanza con facilidad el objetivo gracias al desarrollo basado en \emph{TDD}
    \end{itemize}
    \item Inyector de tráfico HTTP2
    \begin{itemize}
        \item Configurar y lanzar tráfico con un flujo simple (Primera fase)
        \item Inyectar tráfico con un flujo mas completo (Segunda fase)
        \item Ejecución mediante consola de comandos (arquitectura que permita ser extendida)
        \item Configuración del tráfico mediante ficheros \emph{yaml}
        \item Resultados de las ejecuciones por consola de comandos o fichero (arquitectura que permita ser extendida)
    \end{itemize}
    \item Pruebas asociadas
    \begin{itemize}
        \item Pruebas unitarias
        \item Pruebas End to End
    \end{itemize}
    \item Mezclar la metodología \emph{Extreme Programming} con una fase inicial (Sprint-0) de planificación de la arquitectura\cite{design dead}
\end{itemize}

\section{Metodología}
La metodología utilizada será \emph{Extreme Programming} aunque se planifique una arquitectura inicial con el objetivo
de demostrar que se favorece la implementación de las pruebas ya que se facilita el diseño bottom-up.


\newpage
\begin{thebibliography}{Bibliografía}
\bibitem{xp}
Extreme Programming
\\\url{https://www.tutorialspoint.com/extreme_programming/extreme\_programming\_introduction.htm}

\bibitem{cpp}
C++20
\\\url{https://en.cppreference.com/w/cpp/20}

\bibitem{http2}
HTTP2
\\\url{https://http2.github.io/}

\bibitem{nghttp2}
nghttp2
\\\url{https://github.com/nghttp2/nghttp2}

\bibitem{design dead}
Is design dead?
\\\url{https://www.martinfowler.com/articles/designDead.html}
\end{thebibliography}